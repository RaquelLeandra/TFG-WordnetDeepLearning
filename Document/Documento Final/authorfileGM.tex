\documentclass[12,twoside]{TFG-GM}
%\usepackage[active]{srcltx}
\usepackage{amsthm,amsmath,amssymb,amsfonts,amscd}
\usepackage{graphicx}
\usepackage{enumerate}
\usepackage[all]{xy}
\usepackage{booktabs}
%\usepackage[usenames]{xcolor}
%\usepackage{fancyhdr}
\usepackage{minted}
%%%%%Author packages if necessary
\usepackage{hyperref}

% Theorem Environments: add extra ones at the end if you need it.

\newtheorem*{theoremA}{Theorem A}
\newtheorem{theorem}{Theorem}[section]

\newtheorem{proposition}[theorem]{Proposition}
\newtheorem{lemma}[theorem]{Lemma}
\newtheorem{corollary}[theorem]{Corollary}
\newtheorem{conjecture}[theorem]{Conjecture}

\theoremstyle{definition}
\newtheorem{definition}[theorem]{Definition}
\newtheorem{example}[theorem]{Example}

\theoremstyle{remark}
\newtheorem{remark}[theorem]{Remark}
\newtheorem*{remarknonumber}{Remark}
\newtheorem{observation}[theorem]{Observation}




%%%%%%%%%%%%%%%%%%
% macros/abbreviations: Include here your own.
%%%%%%%%%%%%%%%%%%

\newcommand{\N}{\ensuremath{\mathbb{N}}}


% Body of document

\titol{ Wordnet y Deep Learning: Una posible unión}
\titolcurt{Short title}
\authorStudent{Author's full name}
\supervisors{(name of the supervisor/s of the master's thesis)}
\monthYear{Month, year}

%\msc[2010]{Primary  	55M25, 57P10, Secondary 55P15, 57R19, 57N15.}

\paraulesclau{keyword1, keyword2, keyword3, ...}
\agraiments{
Thanks to...}


\abstracteng{This should be an abstract in english, up to 1000 characters.}

%%%%%%%%%
\begin{document}

\maketitle

\section{Estructura del TFG: }

\begin{itemize}
\item \textbf{Introducción (5 pags): }
\begin{itemize}
\item FNN (Explicar neural networks, me puedo inspirar en \url{https://upc-mai-dl.github.io/mlp-convnets-theory/})
\item CNN 
\item EMbeddings  
\end{itemize}
\item \textbf{Related Work (10 pags):}
\begin{itemize}
\item FNE 
\item Wordnet 
\item imagenet 
\end{itemize}
\item \textbf{Approach }
\begin{itemize}
\item Hipotesis iniciales 
\item statistics 
\item insights 
\end{itemize}
\item \textbf{Analysis}
\textbf{DETERMINAR LOS PUNTOS DEL ANÁLISIS PARA EL LUNES 4 }
\end{itemize}

\section{Introducción}

\subsection{Conocimientos previos}
- Explicar el problema inicial (clasificación de imágenes) y por que es difícil.
- Explicar que es una red convolucional 
- Explicar que es transfer learning 
- Explicar el paper An Out-of-the-box Full-network Embedding for
Convolutional Neural Networks
- Explicar que es wordnet

\subsection{Objetivos}
La idea principal de tfg es buscar relaciones entre los synsets de wordnet y el full network embedding. 
Hipótesis iniciales: 
- cuanto más concreto sea el synset más 1 debería tener.
- Cuanto más profundo sea el layer más 1.


\section{Bibliography}


%______________________________________________________________
\appendix
\vfill\newpage \section{Código utilizado}
\begin{minted}{python}
class Data:
    """
    Esta clase consiste en los datos que voy a necesitar para hacer las estadísticas.
    Que no dependen de los synsets elegidos.

    Attributes:
        version (int): versión del embedding que utilizo puede ser 19, 25 o 31
        embedding_path (str): path
        layers (dict): Un diccionario tal que
            layers[string correspondiente al layer] = [inicio del layer, final del layer]

         labels ()

         :parameter version = Version del embedding que utilizo
    """

    def __init__(self, path, version=25):
        """

        :param version: Es la versión del embedding que queremos cargar (25,31,19)
        """
        self.version = version
        _embedding_path = "../Data/vgg16_ImageNet_ALLlayers_C1avg_imagenet_train.npz"
        self.imagenet_id_path = "../Data/synset.txt"
        if version == 25:
            _embedding = 'vgg16_ImageNet_imagenet_C1avg_E_FN_KSBsp0.15n0.25_Gall_train_.npy'
        elif version == 19:
            _embedding = 'vgg16_ImageNet_imagenet_C1avg_E_FN_KSBsp0.11n0.19_Gall_train_.npy'
        elif version == 31:
            _embedding = 'vgg16_ImageNet_imagenet_C1avg_E_FN_KSBsp0.19n0.31_Gall_train_.npy'
        else:
            _embedding = path
            print('No has puesto un embedding válido, usando el de defoult (25)')
        self.discretized_embedding_path = '../Data/Embeddings/' + _embedding
        print('Estamos usando ' + _embedding[-20:-16])
        embedding = np.load(_embedding_path)
        self.labels = embedding['labels']
        # self.matrix = self.embedding['data_matrix']
        del embedding
        self.dmatrix = np.array(np.load(self.discretized_embedding_path))
        self.imagenet_all_ids = np.genfromtxt(self.imagenet_id_path, dtype=np.str)
        self.features_category = [-1, 0, 1]
        self.colors = ['#3643D2', 'c', '#722672', '#BF3FBF']
        self.layers = {
            'conv1_1': [0, 64],  # 1
            'conv1_2': [64, 128],  # 2
            'conv2_1': [128, 256],  # 3
            'conv2_2': [256, 384],  # 4
            'conv3_1': [384, 640],  # 5
            'conv3_2': [640, 896],  # 6
            'conv3_3': [896, 1152],  # 7
            'conv4_1': [1152, 1664],  # 8
            'conv4_2': [1664, 2176],  # 9
            'conv4_3': [2176, 2688],  # 10
            'conv5_1': [2688, 3200],  # 11
            'conv5_2': [3200, 3712],  # 12
            'conv5_3': [3712, 4224],  # 13
            'fc6': [4224, 8320],  # 14
            'fc7': [8320, 12416],  # 15
            'conv1': [0, 128],  # 16
            'conv2': [128, 384],  # 17
            'conv3': [384, 1152],  # 18
            'conv4': [1152, 2688],  # 19
            'conv5': [2688, 4224],  # 20
            'conv': [0, 4224],  # 21
            'fc6tofc7': [4224, 12416],  # 23
            # 'all':[0,12416]          # 24
        }
        self.reduced_layers = {
            'conv1': [0, 128],
            'conv2': [128, 384],
            'conv3': [384, 1152],
            'conv4': [1152, 2688],
            'conv5': [2688, 4224],
            'fc6': [4224, 8320],
            'fc7': [8320, 12416]
        }

    def __del__(self):
        self.embedding = None
        self.dmatrix = None
        self.version = None
        self.embedding_path = None
        self.layers = None
        self.labels = None
        self.features_category = None
        self.colors = None
        gc.collect()


\end{minted}


\end{document}


